\documentclass[a4paper, 12pt]{article}
\usepackage[slovene]{babel}
\usepackage[utf8]{inputenc}
\usepackage{lmodern}
\usepackage[T1]{fontenc}
\usepackage{eurosym}

\begin {document}

\begin{center}
{\huge\textbf{Imena}}\\
{\large{Projekt pri predmetu Programiranje 1}}\\
{\large\textsc{Katarina Černe, Neža Dimec}}
\end{center}

\tableofcontents

\section{Opis projekta}

Aplikacija, ki je bila ustvarjena pri projektu Imena, omogoča uporabniku, da preko enostavnega iskalnika izbrska nekaj statističnih podatkov o vnešenem imenu:

\begin{itemize}
\item število pojavitev določenega imena v Sloveniji glede na spol,
\item pogostost v Sloveniji,
\end{itemize}

\noindent ter poleg izbrska še podatke o:

\begin{itemize}
\item pomenu imena,
\item izvoru imena,
\item izvorni obliki imena,
\item godovnem dnevu.
\end{itemize}

V primeru, ko se poleg izpisanega rezultata pojavi še *, obstaja možnost, da izpisan rezultat ni najbolj natančen, saj je aplikacija zaradi pomanjkljive spletne baze za vnešeno ime, podatke iskala preko imena, ki je zapisano v izvorni obliki prvotnega imena.

\section {Podatkovni viri}

Vir podatkov sta spletni strani:
\begin{itemize}
\item http://www.stat.si/imena.asp\\
(vir statističnih podatkov)
\item http://sl.wikipedia.org/w/index.php?\\
(vir podatkov o pomenu, izvoru, izvorni obliki, godovnem dnevu)
\end{itemize}

\section{Uporabniški vmesnik}

Program  "gui\_demo.py"~ ob zagonu odpre okno, kamor lahko v iskalno okence vpišemo željeno poizvedbo. Pod iskalnim okencem se nam ponuja izbira spola. Iskanje poženemo z gumbom "Išči!"~ ali s pritiskom na tipko "Enter". Iskalna funkcija nam nato vrne iskane podatke in jih izpiše uporabniku v prikazovalniku spodaj.

\section{Testiranje}
Program za iskanje sva testirali z naključno izbranimi imeni, zato ne izključujeva možnosti, da morda za kakšno izmed imen aplikacija ne prikaže vseh željenih podakov, v primeru, da je stran o imenu iz wikipedije oblikovana kako drugače kot v običajnem primeru. Na primerih, ki sva jih uspeli preizkusiti, aplikacija deluje pravilno.






\end{document}